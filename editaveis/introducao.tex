\section{Contextualização}

Uma das grandes demandas de profissionais e entusiastas do esporte de ciclismo é ter a opção de poder praticar em um ambiente fechado, seja em casa, academia ou laboratório. Este interesse é natural por questões de praticidade, comodidade ou por questões mais sérias, tais como avaliações de desempenho. Para este intuito são usadas plataformas de ciclismo estáticas, onde o atleta pode desempenhar sua atividade restrita a um espaço pequeno e fechado. Um exemplo de plataforma estática que ganhou muita fama é a bicicleta ergométrica.

Contudo, o fator preponderante nesta aplicação é o quão próximo os estímulos que essa plataforma estática de ciclismo causa estão dos estímulos de uma corrida de bicicleta em ambiente livre. Neste quesito entram as três grandes propostas deste projeto, que são a plataforma estática iterativa, imersão em ambiente de realidade virtual e medição de dados fisiológicos e de desempenho.

\section{Objetivo Geral}

	Construção de uma plataforma de ciclismo estática e iterativa com imersão do usuário em ambiente de realidade virtual com uso de óculos de realidade virtual e monitoramento e armazenamento de dados fisiológicos e de desempenho.

\section{Objetivos Específicos}

        	Para atingir o objetivo geral e atender a demanda exigida pelo projeto deverão ser cumpridos os seguintes objetivos específicos:

\begin{itemize}
\item Projeto e construção da plataforma de acoplamento para bicicleta.

\item Projeto do sistema de alimentação dos atuadores e sistemas de sensoriamento.

\item Implementação de um sistema de geração de energia para a plataforma.

\item Desenvolvimento do sistema de controle dos componentes.

\item Projeto dos sistemas de sensoriamento e construção dos circuitos.

\item Configuração do microcontrolador responsável pela aquisição e transmissão de dados.

\item Desenvolvimento de ambiente e jogo de realidade virtual.

\item Estabelecimento de um protocolo de comunicação entre os sistemas de aquisição e processador central (PC).

\item Armazenamento de informações adquiridas pelos sensores em um registro.
\end{itemize}